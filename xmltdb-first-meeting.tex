\documentclass[12pt]{article}
\usepackage[utf8]{inputenc}
\usepackage{amssymb}
\usepackage{graphicx,subfigure}              % with figures
% sometimes needed to have pdf files 
\pdfsuppresswarningpagegroup=1
\topmargin -1mm
\oddsidemargin -1mm
\evensidemargin -1mm
\textwidth 170mm
\textheight 220mm
\parskip 2mm
\parindent 3mm
\pagestyle{empty}
\usepackage{xcolor}
\usepackage[normalem]{ulem}

% For appendices
%\usepackage[titletoc,title,header]{appendix}

%\usepackage[firstpage]{draftwatermark}
%\SetWatermarkScale{4}

\begin{document}

{\Large \bf Agenda for meeting 4-5 May 2023 in Udine}

\bigskip

The 4 May the meeting will begin 13.00 (UTC+1, Tokyo 21.00; New York 07.00)

Maybe Italy has summertime, i.e. UTC+2, one hour earlier in Tokyo and New York.

% UTC+9   UTC+1  UTC-5
%  +8       0      -6
% Tokyo   Udine  New York
% 21.00   13.00  07.00
%         UTC+2  summer time?
% 20.00   13.00  06.00

%\bigskip

\begin{enumerate}
\item 13.00 Welcome and a short presentation by each participant.

\item 13.30 An attempt to summarize the important tasks of a database
  manager: Participating in assessments, collecting published
  assessments, testing extrapolations, estimating missing parameters,
  updating old assessments, incorporate feedback from database users,
  information useful for future managers.  What can be documented in
  the database or need additional files/figures?  All are welcome to
  prepare a few slides!

\item 14.30 Coffee break
  
\item 14.50 Details of the XML format and how it can be adapted for
  TDB files to improve the documentation and consistency.

\item 15.10 Survey of old and new models used in the TDB file and how
  they can be handled in the XML file.

\item 15.30 A simple independent software to upload a TDB file to a
  rudimentary XMLTDB format.

\item 16.00 Discussion how we should organize the project with the
  limited funding.  Tasks for each participant. Next meeting.

\item 17.00 End of the first day.
\end{enumerate}

Second day 5 May the meeting will start 09.00 (bad time for TEAMS,
probably only particpants in Europe)

\begin{enumerate}
\item 09.00 Economy.

\item 09.30 New unary models and parameters.

\item 10.30 Coffee break

\item 10.50 How to handle particular models in the XMLTDB file.

\item 11.10 Structurng model parameter identifiers including
  non-thermodynamc properties.

\item 11.30 Free discussion

\item 12.00 End of meeting.

\end{enumerate}

\newpage

{\bf \large Some ideas of organizing the database management}

I have (again) summarized some ideas here and feedback from all users
and developers are important.  You are all welcome to circulate your
opinon of what to discuss prepare a short presentation of how to
manage a thermodynamic database for the meeting 4-5 May.

A TDB file consists of data for elements, species and phases.  All
thermodynamic data is associated with the phases and the models for
the phases.  The current TDB format have a number of features that can
be improved, in particular how to describe different model features.
For example the {\bf TYPE\_DEFINITION} keyword should be replaced by
an {\bf AMEND} specifying for each phase some particular model
feature.

I want again stress that the XMLTDB file will normally not be read
directly by a thermodynamic software, nor by the database manager.
There will be an ``upload'' and``download'' software to convert an
XMLTDB file to the software specific TDB file format and vice versa.
That was the original idea with TDB the format.

Some points (several already stated):
\begin{enumerate}
\item An XMLTDB file will normally not be edited directly which means
  there must be efficient software to upload/download this format to
  the software specific TDB format. A main objective is that the
  database managers must be able to add critical information about the
  models and parameters used in a database.  Such information is very
  important for the second and later generation of database managers.

  There must be efficient means for ``comments'' in the ``master TDB''
  file edited by the database manager to be included in the XMLTDB
  file.  These comments my be supressed on a TDB file provided to a
  user.

\item Different database managers have different ways of organize
  their ``master TDB'' file and it should be possible to upload and
  download the XMLTDB file in their preferred way including all their
  comments.  For the general user the data are normally ordered by
  phase but for updating it may be better to have them ordered as
  unary, binary, ternary and higher order systems.  But it would be
  interesting to know how each manager handles the database and if
  there are some useful ideas to share.

\item Some model features have different implementation in various
  software but the way to handle them in the XMLTDB file should be
  independent of a particular implementation.  When downloading a TDB
  file from an XMLTDB file the downloading software should be able to
  handle such differences.
  
\item The ``model parameter identifier'' (MPI) in a parameter
  associates the parameter with some model for the phase.  A parameter
  with the MPI {\bf G} is associated with the fundamental Gibbs
  energy function, {\bf TC} and {\bf BMAG} are associated with
  different magnetic models.

  Several new MPI have also been added by the new unary models and by
  different software, also for handling data that are not associated
  with the thermodynamic data, for example variants of {\bf MQ} for
  mobility data.  Some of these may be unique for different
  constituents or components.  We should try to establish an agreed
  set and format for the MPI.

  A parameter is a function of $T$ and $P$ and has a reference to the
  phase and the constituents, the fraction of which should be
  multiplied with the parameter.  This makes it possible for a
  parameter to appear anywhere in a TDB file (or at least anywhere
  after the phase and its constituents have been entered).  But each
  database manager may have his/her own preference.
  
\item It should be checked that particular MPI parameters for a phase
  have the appropriate model associated with the phase when uploading
  a TDB file.  Also check that a phase with a model requiring a
  particular MPI have such parameters, unless the model will give zero
  contribution with no such parameters.  I am not sure if the new
  magnetic model does that.

\item Originally the TDB file was read once sequentially but new
  models means that several rewind/read are often needed, in
  particular to pick up missing functions and ``disordered parts'' of
  phases.  We should consider this when designing the download
  software.

\item The ``disordered part'' or ``partitioned'' model is an important
  simplification to handle parameters for phases with several
  sublattices.  In the original TC implementation this was achieved by
  adding two separate phases, in OC there is a single phase with two
  sets of constituent set of fractions where the disordered fractions
  are calculated from the ordered set.  This makes it easier for a
  manager to handle the model parameters.

  In some cases only the disordered part of an FCC phase is needed but
  it is better to keep the ordered and disordered parts together in
  the XMLTDB file.  It can be an option when downloading to a TDB file
  to ignore the constituents and parameters for the ordered part.

  In other cases the disordered part is useful to reduce the number of
  parameters, for example in a 5 sublattice model of $\sigma$ for a
  multicomponent system there many parameters G(sigma,A:B:C:D:E) which
  are not assessed and described by\\

  PARAMETER G(SIGMA,A:B:C:D:E) 298.15 2*GSIGMA\_A+4*GSIGMA\_B\\
  +8*GSIGMA\_C+8*GSIGMA\_D+8*GSIGMA\_E; 6000 N!

  Such parameters can simply be ignored if there is a disordered
  fraction set for A, B, C, D, E with the pure element parameters.
  Only assessed ordered parameters are needed.

\item As proposed by Nathalie we should no longer use the model (for
  phases with order/disorder transformation) where the ordered part of
  a partitioned phase is subtracted using the disordered set of
  fractions in the ordered aublattices, i.e.
  \begin{eqnarray}
    G_M &=& G^{\rm disord}_M(x) + (G^{\rm ord}_M(y) - G^{\rm ord}_M(y=x))
  \end{eqnarray}
  where $G^{\rm disord}_M(x)$ is the Gibbs energy for the disordered
  part, $G^{\rm ord}_M(y)$ is that for the ordered part and
  $G^{\rm~ord}_M(y=x)$ is the ordered part recalculated as disordered.
  The latter two cancel in the disordered state.  We should simply
  add the ordered and disordered parts:
  \begin{eqnarray}
    G_M &=& G^{\rm disord}_M(x) + G^{\rm ord}_M(y)
  \end{eqnarray}

  The only advanatage of subtracting the ordered part ``as disordred''
  is that the disordered part can be assessed independently of the
  ordered one but there are few systems where that is useful.
  
\item The ``parameter permutation'' feature is interesting for phases
  with order/disorder transformation.  It can reduce significantly the
  number of parameters as all parameters\\
  G(FCC\_4SL,A:A:A:B)\\
  with the B in different sublattices are the same.  Preferably they
  occur only once in the XMLTDB file and are duplicated when
  downloaded to a TDB file for a software in which permutations are
  not implemented.

\item We have several models for the liquid and it might be
  interesting to have duplicate models of the same phase in the XMLTDB
  file because the other phases may be the same.  At least a regular
  solution liquid phase in parallel with the I2SL model which in many
  cases is identical.  It should be possible when downloading a TDB
  file to select the model for such a phase.

  As already mentioned an ordered or disordered version of FCC and BCC
  phases may be selected.  But they would not require duplicate set of
  parameters in the XMLTDB file.

\item We must consider the models developed by the new unary project
  such as the low $T$ vibrational entropy, new magnetic reference
  state, the two-state liquid model, the Equivalent Entropy Criteria
  (EEC).  This has created new types of MPI.

  It has been decided not to use multiple composition dependent
  Einstein-$T$ describing the heat capacity of a pure element down to
  0~K.  I am very much in favour of this but such a feature may be
  useful for oxides and some other kinds compounds.  Multiple
  Einstein-$T$ parameters for the unaries would create confusion for
  assessments.
  
\item The elements and species are simple to handle except that
  species has a problem handling the MQMQA and UNIQAC models.  In
  these models the species have additional properties needed such as
  bonds and FNN/SNN ratios or volume and area needed to calculate the
  configurational entropy.

\item A favourite topic for me has been the names of phases.  But I
  think we have reached a reasonable agreement using a ``popular''
  name such as FCC augemnted with a ``Structurbericht'' notation, A1,
  for austenite.  Although it is a complication that for example TiC
  is the same phase.

\item An important point is also to convince journals that all
  published assessments should provide the assessed parameters in an
  XMLTDB file as supplementary data.

\end{enumerate}

\end{document}

\begin{enumerate}
\item We should try to establish what we mean with different models.
  For me I consider
  \begin{enumerate}
  \item The IDEAL model with a single set of sites and no excess parameters.
    Typically the gas phase at high $T$ and low $P$.
  \item A REGULAR model with a single set of sites and excess parameters.
  \item A CEF (Compound Energy Formalism) based model has normally two
    or more sublattices with ideal mixing on each.  Excess parameters
    for interactions on each sublattices and also.  The ideal and
    regular models can be considered as  special cases of CEF.
  \item The MQMQA model as developed by Pelton et al.
  \item The cell model developed by Guy and 
  \end{enumerate}
\end{enumerate}


  \begin{eqnarray}
\end{eqnarray}

\begin{enumerate}
\end{enumerate}
