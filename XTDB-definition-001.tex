\documentclass{article}
\usepackage{graphicx} % Required for inserting images
%\usepackage{geometry}
%\geometry{margin=0.5in}
%\usepackage{cite}
%\usepackage{color}

%\title{XMLTDB01}
%\author{Bo Sundman et al. }
%\date{June 2023}

\usepackage[utf8]{inputenc}
\usepackage{amssymb}
\usepackage{graphicx,subfigure}              % with figures
% sometimes needed to have pdf files 
\pdfsuppresswarningpagegroup=1
\topmargin -10mm
\oddsidemargin -1mm
\evensidemargin -1mm
\textwidth 170mm
\textheight 220mm
\parskip 2mm
\parindent 3mm
%\pagestyle{empty}
\usepackage{xcolor}
\usepackage[normalem]{ulem}

%footnote using capital letter 
%\renewcommand{\thefootnote}{\Alph{footnote}}
%\renewcommand{\thefootnote}{{\arabic{footnote}}

% For appendices
\usepackage[titletoc,title,header]{appendix}

\begin{document}

\begin{center}

  {\Large \bf A proposal of an XML format for TDB files}

  Bo Sundman et al. \today
\end{center}

It seems there is a general agreement that an XML format is beneficial
for Calphad databases, the discussion is how detailed it should be.
The proposal here is intended to make it possible to edit an XTDB file
as easily as the current TDB format while providing better facilities
to handle new features implemented in the different softwares using
TDB files.  This may require modifications in the XML format already
adoped in some software.

\section{XTDB version 0.0.1}

The following proposal is for a common definition of an XTDB file
format based on XML.  It can hopefully become an SGTE standard
acceptable by all software which develop or use Calphad databases.

As XML is extendable each software may add their own flavour in a
controlled way and it will be easier to convert an XTDB file from one
software to another.

The ``note'' for the attributes for an XML element in the tables below
is ``+'' for a compulsory attribute, ``--'' for an optional one or a
letter A-Z for special cases explained in section~\ref{sec:notes}.

All {\bf Id} attributes are case insenitive as in the TDB file.
Except for phase names the ``Id'' cannot be abbreviated when used in
attributes.

An XML parser is convenient for reading XML files but to simplify the
life for the humans reading the XTDB file or software using sequential
read operations, it is recommended that the attributes to an XML
element are provided in the order listed for each element below.

The order of XML elements is free but to simplify for a human reading
and editing the XTDB file a recommended order is {\bf Defaults,
  DatabaseInfo, SoftSpecific} followed by {\bf Element, Species,
  Model} and {\bf Phases} and then {\bf TPfun}s and {\bf Parameter}s
for the pure elements.  After that {\bf TPfun} and {\bf Parameter} for
binary, ternary and higher order systems, normally arranged for each
subsystem.

An XML element referenced in the text below is written in {\bf bold}.

\subsection{The system XTDB elements}

\bigskip
\begin{tabular}{|p{0.1\textwidth} p{0.1\textwidth} r p{0.65\textwidth}|}\hline
  Element & Attributes & Note & Explanation\\\hline

  XTDB    &$>>>$&& Containing XML elements for an XTDB database.\\
          & Version & + & Version of XTDB for this file.\\
          &Software & + & Name of software generating the database.\\
          &Date     & + & Year/month/day the database was written or edited\\
          &Signature& -- & Name/email of person or organisation generating the database.\\\hline
  
  Defaults &$>>>$&& Provides default values of attributes in other XML elements.\\
           & LowT & -- & Default value of low $T$ limit.\\
           & HighT & -- & Default value of high $T$ limit.\\
           & Refid & -- & Default bibliographic reference. \\
           & Elements & -- & VA and/or /- (the electron).\\
           & Model & -- & Any model applicable for the system, for example EEC\\\hline

  DatabaseInfo &$>>>$&& Free text.\\
  & Date & -- & Last update.\\\hline

  SoftSpecific &$>>>>$&& Free text explaining software specific XML
  elements and attributes used.\\\hline
  
\end{tabular}

\subsection{The chemical element and species element}

\bigskip
\begin{tabular}{|p{0.1\textwidth} p{0.1\textwidth} r p{0.65\textwidth}|}\hline
  Element & Attributes & Note & Explanation\\\hline

  Element &$>>>$&& Specifies a chemical element in the database.  The vacancy is denoted ``VA'' and the electron ``/-''.\\
          & Id        & +    & Chemical element symbol, for example FE, H \\
          & Refstate  & +    & Name of a phase, for example GAS \\
          & Mass      & +    & Mass in g/mol\\
          & H298      & --    & Enthalpy difference between 0 and 298.15~K in reference state.\\
          & S298      & --    & Entropy difference between 0 and 298.15~K in reference state.\\\hline

  Species &$>>>$&& Specifies a molecular like aggregate used as constituent of phases.  The elements, except the electron, are also species.\\
          & Id        & A & Species name max 24 letters. For note see \ref{sec:notes}.\\
          & Stoichiometry & B & One or more elements each followed by an unsigned real representing the stoichiometric ratio.  For note see \ref{sec:notes}.  See also appendix~\ref{sec:elementexample}.\\
          & MQMQA & C & For a constituent in the MQMQA model.  For note see \ref{sec:notes}.\\
          & UNIQUAC & D & For a constituent in the UNIQUAC model.  For note see \ref{sec:notes}.\\\hline

\end{tabular}

\subsection{The function and temperature range elements}

There is at present no way to handle several pressure ranges.  The
value of a function, as well as its first and second derivative with
respect to $T$, must be continuous across a change of $T$ range.

\bigskip
\begin{tabular}{|p{0.1\textwidth} p{0.1\textwidth} r p{0.65\textwidth}|}\hline
  Element & Attributes & Note & Explanation\\\hline

  TPfun &$>>>$&& Defines a $T, P$ expression to be used in parameters or other functions.\\
        & Id & E & Function name, max 16 characters. The name can be used in the ``Expr'' attribute of other functions or parameters.  For note see \ref{sec:notes}.\\
        & LowT & -- & If the default low $T$ limit applies.\\
        & Expr & + & Simple mathematical expression terminated by ;.   Use the {\bf Trange} element if several ranges.  See section \ref{sec:expr}\\
        & HighT & -- & If the default high $T$ limit applies.\\\hline

  Trange &$>>>$&& Only inside a {\bf TPfun} or {\bf Parameter} element for an expression with several $T$ ranges.\\ 
         & Expr & + & Simple mathematical expression terminated by ;.  See section \ref{sec:expr}.\\
         & HighT & -- & If the default high $T$ limit applies.\\\hline
\end{tabular}

\subsection{The phase element}\label{sec:phase}

Normally the software requires information of sublattices,
constrituents and models for the phase in order to create the
appropriate datastructures.  The {\bf Parameter} and some models, see
section~\ref{sec:toop} can appear anywhere in the XTDB file.

The {\bf Splitphase} model (also known as the disordered fraction
model) combines the Gibbs energies of an ordered and a disordered
phase where the disordered phase has fewer sublattices than the
ordered and the constituent fractions on its first sublattice are the
average of the fractions summed over several sublattices in the
ordered phase as specified by the attribute {\bf Sum}.  The {\bf
  SplitPhase} element must appear within the ordered {\bf phase}
element before any other nested XML elements such as {\bf AmendPhase,
  Parameter}, {\bf KohlerModel} or {\bf ToopModel} elements.  Tho {\bf
  Parameter, ToopModel} and {\bf KohlerModel} elements may appear anywhere in
the XTB file.

Some software may not have a separate disordered phase but arrange all
{\bf Parameter} elements for the ordered and disordered phase within
the same phase.

\begin{tabular}{|p{0.1\textwidth} p{0.1\textwidth} r p{0.65\textwidth}|}\hline
  Element & Attributes & Note & Explanation\\\hline

  Phase &$>>>$&& All data belongs to a phase.\\
       & Id & F & Phase name, see \ref{sec:notes} note F. \\
       & Configuration & G & Model for the configurational entropy.  For note see \ref{sec:notes}.\\
       & State & -- & G for gas phase, L for liquid phase.  Needed if EEC is used.\\\hline

  Sublattices &$>>>$&& Only inside a {\bf Phase} element.\\
        & Number-of & + & Number of sublattices, an integer value 1-10 \\
        & Ratios & + & Sites on the sublattices, as many reals as sublattices separated by a space.\\\hline

  Constituents &$>>>$&& Only inside a {\bf Phase} element.\\
        & Sublattice & + & Indicates the sublattice for the list of species.\\
        & List & + & Species Id, separated by a space, in the sublattice.\\\hline

  SplitPhase &$>>>$&& Only inside the {\bf Phase} element of the ordered phase.\\
      & Refid & + & Where the model is described.\\
%      & Ordered & J & May be omitted, for note see \ref{sec:notes}.\\
      & Disordered & K & Name of disordered phase, for note see \ref{sec:notes}.\\
      & Sum & + & Number of sublattices in the ordered phase to be summed for the disordered phase constitution.  All sublattices summed must have the some constituents.\\
      & Subtract & L & Must have value Y if the Gibbs energy of ordered phase as disordered should be subtracted, for note see \ref{sec:notes}.\\\hline

  AmendPhase &$>>>$&& Only inside a {\bf Phase} element.\\
  & Model & + & List of models ``Id'', separated by a space, for this phase.  The ``Id'' is specified in a model element below.\\\hline
\end{tabular}

%\newpage
%%%%%%%%%%%%%%%%%%%%%%%%%%%%%%%%%%%%%%%%%%%%%%%%%%%%%%%%%%%%% page 2

\subsection{The simple parameter element}

All thermodynamic data are defined by the parameter elements.  They
can be arranged inside a phase element or separately for each binary,
ternary etc. subsystem.  See also section~\ref{sec:parameter2} and
  appendix~\ref{sec:parameter examples}.

\bigskip
\begin{tabular}{|p{0.1\textwidth} p{0.1\textwidth} r p{0.65\textwidth}|}\hline
  Element & Attributes & Note & Explanation\\\hline

  Parameter &$>>>$&& Specifies the $T, P$ expression of a model parameter for a set of constituents.\\
      & Id & + & As in TDB file, for example G(LIQUID,A,B:VA;2).  See also the {\bf Parameter2} element.  May include the {\bf Trange} element for parameters with multiple ranges.\\
      & LowT & -- & If the default low $T$ limit applies.\\
      & Expr & + & Simple mathematical expression terminated by ;.  If several ranges use a {\bf Trange} element.  See section \ref{sec:expr}.\\
      & HighT & -- & If the default high $T$ limit applies.\\
      & Bibref & + & Bibliographic reference.\\\hline
\end{tabular}

\subsection{The elaborate parameter element}\label{sec:parameter2}

An alternative more elaborated XML element can be used for parameters
which may be preferred by software.  It is striaghtforward to convert
from one to another.

\bigskip
\begin{tabular}{|p{0.1\textwidth} p{0.1\textwidth} r p{0.65\textwidth}|}\hline
  Element & Attributes & Note & Explanation\\\hline

  Parameter2 &$>>>$&& A more detailed parameter element preferred by software.\\
      & Id & -- & As in TDB file, for example G(LIQUID,A,B:VA;2).  See also the {\bf Parameter} element. \\
      & MPID & H & Model parameter identifier, for example G or TC.  For note see \ref{sec:notes}.\\
      & Phase & -- & Can be omitted inside a phase element, otherwise the phase name.\\
      & LowT & -- & If the default low $T$ limit applies.\\
      & Expr & + & Simple mathematical expression terminated by ;.  If several ranges use a {\bf Trange} element.  See section \ref{sec:expr}.\\
      & HighT & -- & If the default high $T$ limit applies.\\
      & Bibref & + & Bibliographic reference.\\\hline

  ConstArray &$>>>$&& Only inside a {\bf parameter2} element.\\
      & Degree & I & Can be omitted if zero, for note see \ref{sec:notes}.\\\hline
  
  Constituent &$>>>$&& Only inside a {\bf ConstArray} element.\\
      & Sublattice & + &  Sublattice of constituent.\\
      & Species & + & Name of a constituent.\\\hline
\end{tabular}

\bigskip
The {\bf Parameter2} element is preferred by software but for manual
editing {\bf Parameter} is simpler, see the
appendix~\ref{sec:parameter examples}.  Both elements can be used in
the XTDB file.  Software can convert a {\bf Parameter} element to a
{\bf Parameter2} element.  If an ``Id'' is present in a {\bf
  parameter2} element the software should check that the ``Id'' is
identical to the long form and report an error if not.

The use of wildcards, ``*'', for species in a parameter must be
defined, see section~\ref{sec:wildcard}.


%\newpage

\subsection{Bibliography for parameters and models.}\label{sec:biblio}

For each parameter there must be a bibliographic reference.  This can
be the same for all parameters from the same source.  The models also
have a reference which should indicate a paper where the model is
explained, see Appendix~\ref{sec:modelex}.

\bigskip
\begin{tabular}{|p{0.1\textwidth} p{0.1\textwidth} r p{0.65\textwidth}|}\hline
  Element & Attributes & Note & Explanation\\\hline

  Bibliography &$>>>$&& Contains bibliographic references\\\hline

  Bibitem &$>>>$&& Only inside a {\bf Bibliography} element.\\
      & Id & + &   Used as bibref attribute for a parameter or model, normally a paper or a comment by the database manager.\\
      & Text & + & Reference to a paper or comment\\
      & Date & -- & Date when the bibref element was created\\
      & Sign & -- & Signature of the one adding the reference\\\hline

\end{tabular}

%\newpage
%%%%%%%%%%%%%%%%%%%%%%%%%%%%%%%%%%%%%%%%%%%%%%%%%%%%%%%%%%%%%%%%%%%

\subsection{XTDB model elements}\label{sec:models}

These XML elements for generally accepted models can be on a separate
file but the model ``Id'' is the important and used in the {\bf
  AmendPhase} element for each phase which has the model.  Most model
has one or more model parameter identifiers (MPID) used in the {\bf
  Parameter} element.  See also appendix~\ref{sec:modelapp}.  Software
specific models should be explained in this XML element.

\bigskip
\begin{tabular}{|p{0.1\textwidth} p{0.1\textwidth} r p{0.65\textwidth}|}\hline
  Element & Attributes & Note & Explanation\\\hline

  Models   &$>>>$&& Contains model descriptions and identifiers used in ``AmendPhase'' elements for phases.  The models usually specify one or more model parameter identifiers needed by the model.\\\hline
\end{tabular}

The following XML elements should only be used inside a {\bf Model} element
and are used to specify the model parameter identifiers associated with the
model.

\subsubsection{XTDB generally accepted model elements}

\bigskip
\begin{tabular}{|p{0.1\textwidth} p{0.1\textwidth} r p{0.65\textwidth}|}\hline
  Element & Attributes & Note & Explanation\\\hline

%  Models   &$>>>$&& Contains model descriptions and identifiers used in ``AmendPhase'' elements for phases.  The models usually specify one or more model parameter identifiers needed by the model.\\\hline

  Magnetic &$>>>$&& There are several magnetic models.\\
      & Id & + & This Id is used in {\bf AmendPhase} element.\\
      & Refid & + & Where the model is described.\\
      & MPID1 & + & Specifies a magnetic model parameter identifier (MPID) for parameters.\\
      & MPID2 & + & Specifies a magnetic model parameter identifier (MPID) for parameters.\\
      & MPID3 & + & Specifies a magnetic model parameter identifier (MPID) for parameters.\\\hline

  Volume &$>>>$&& Specifies the model for volume of a phase.\\
      & Id & + & This Id is used in the {\bf AmendPhase} element.\\
      & Refid & + & Where the model is described.\\
      & MPID1 & + & Specifies volume model parameter identifier (MPID) for parameters.\\
      & MPID2 & + & Specifies volume model parameter identifier (MPID) for parameters.\\
      & MPID3 & + & Specifies volume model parameter identifier (MPID) for parameters.\\\hline

  FCC4Perm &$>>>$& & For the FCC 4 sublattice tetrahedron model identical permutations of a parameter are listed only once.\\
     & Id & + & This Id is used in {\bf AmendPhase} element.\\
     & Refid & + & Where the model is described.\\\hline

  BCC4Perm &$>>>$& & For the BCC 4 sublattice tetrahedron model identical permutations of a parameter are listed only once.\\
     & Id & + & This Id is used in {\bf AmendPhase} element.\\
     & Refid & + & Where the model is described.\\\hline

\end{tabular}

\subsubsection{New unary models}

There is also a new magentic model.

\bigskip
\begin{tabular}{|p{0.1\textwidth} p{0.1\textwidth} r p{0.65\textwidth}|}\hline
  Element & Attributes & Note & Explanation\\\hline

  Einstein &$>>>$&& The low $T$ vibrational model.\\
      & Id & + & This Id is used in {\bf AmendPhase} element.\\
      & Refid & + & Where the model is described.\\
      & MPID1 & + & Specifies the Einstein model parameter identifier (MPID) for parameters.\\\hline

  Liquid2state &$>>>$&& The liquid 2-state model.\\
      & Id & + & This Id is used in {\bf AmendPhase} element.\\
      & Refid & + & Where the model is described.\\
      & MPID1 & + & Specifies liquid model parameter identifier (MPID) for parameters.\\
      & MPID2 & + & Specifies Einstein model parameter identifier (MPID) for parameters.\\\hline

  EEC &$>>>$&& Specifies that the Equi-entropy model applies to current database\\
      & Refid & + & Where the model is described.\\\hline

\end{tabular}

\subsubsection{XTDB elements for constituent specific models}\label{sec:toop}

These models are related to specific constituents of a phase and they
must be specified explicitly and cannot be included in the {\bf
  AmendPhase} element for the phase because they include additional
information.  The Toop and Kohler ternary extrapolation models are
most likely specified together with the parameters for a specific
ternary.  The Muggianu ternary model is the default extrapolation
model.

\bigskip
\begin{tabular}{|p{0.1\textwidth} p{0.1\textwidth} r p{0.65\textwidth}|}\hline
  Element & Attributes & Note & Explanation\\\hline

  ToopModel &$>>>$& & Subset of 3 constituents for which the Toop ternary extrapolation model should be used.    By default the Muggianu extrapolation is used.\\
      & Refid & + & Where the model is described.\\
      & Phase & + & Can be omitted if used inside a {\bf Phase} element.  Otherwise the phase for which the model should be used.\\
      & Constituents & + & Specifies 3 constituents, the Toop constituent first.\\\hline

  KohlerModel &$>>>$&  & Subset of 3 constituents for which the Kohler ternary extrapolation model should be used.  By default the Muggianu extrapolation is used.\\
      & Refid & + & Where the model is described.\\
      & Phase & + & Can be omitted if used inside a {\bf Phase} element.  Otherwise the phase for which the model should be used.\\
      & Constituents & + & Specifies 3 constituents in any order.\\\hline

\end{tabular}

\subsection{Notes}\label{sec:notes}

Note that the XML elements and attributes are case sensitive whereas
the names of elements, species, functions and other text are case
insensitive.  For example a chemical element written FE, Fe, fe and fE
is the same element.  Thus CO is Cobalt and C1O is carbonmonoxide.

\begin{description}
\item{\bf A:}\label{sec:noteA} A species name or {\bf Id} must start
  with letter A-Z and can contain letters, digits and the special
  characters ``\_'', ``/'', ``-'' and ``+''.  It must not be
  abbreviated when used as constituent in a {\bf Phase} or {\bf
    Parameter} element.

  Should the characters ``-'' and ``\_'' be treated as identical(?)

\item{\bf B:} The species stoichiometry is a sequence of one or more
  element names (case insensitive) followed by a real number
  specifying the stoichiometric ratio.  Following the TDB standard an
  element with a two letter name does not need a stoichiometric ratio
  equal to unity, for example MGO for MG1O1.  A final stoichiometry
  unity can also be ignored.  No parenthesis are allowed.  For MQMQA
  clusters the stoichiometry can be omitted.

\item{\bf C:} The MQMQA attribute should contain two or more element
  names separated a comma or a sublattice separator ``:'' and
  followed by equal number of unsigned reals representing the bond
  fractions, For example:

  CS,LA:F  9.0 6.0 4.0

  See also appendix~\ref{sec:elementexample}.  According to the MQMQA
  model the cluster stoichometry is calculated as 6.0 divided by the
  bond fractions.  The cluster is always electrically neutral.  For
  endmember clusters (with one species in each sublattice) a real
  representing the SNN/FNN ratio, usually 2.4, must be supplied.

\item{\bf D:} For UNIQUAC species.  Contains two reals representing
  volume and area of the species in m$^3$ and m$^2$ (?).

\item{\bf E:} A {\bf TPfun} name must start with a letter A-Z and may
  contain letters, digits and the special character ``\_''.  It must
  not be longer than 16 characters.  It cannot be abbreviated.

\item{\bf F:} A phase name must start with a letter A-Z and have no
  more than 24 characters.  It can contain letters, numbers and the
  special character ``\_''.  In some software``(`` and ``)'' may be
  used in phase names but they can be converted to ``\_'' in other
  software.

  A phase name may be abbreviated in parameters and some other cases
  and thus a phase name must be unique and not an abbreviation of
  another phase.

\item{\bf G:} The configuration model, for example CEF, I2SL, MQMQA
  etc.  Maybe also``RKM'' for a Redlich-Kister-Muggianu model with a
  single lattice and one site?

\item{\bf H:} An initial set of model parameter identifies (MPID) must
  be defined, see section~\ref{sec:mpid} and appendix~\ref{sec:mpid2}.
  At present G is the Gibbs energy, TC Curie $T$, BMAGN the Bohr
  magneton number etc.  The MPID must not be abbreviated.

\item{\bf I:} The degree can be omitted if zero.

\item{\bf K:} The disordered phase within a {\bf SplitPhase} element
  can be omitted if the parameters for the disordered phase are
  provided within the ordered phase, using a suffix ``D'' after the
  MPID name and with the appropriate reduced number of sublattices.
  
  For example GD(SIGMA,CR) can be used as {\bf Id} in a {\bf
    Parameter} element for the lattice stability of Cr in SIGMA if the
  SIGMA phase is modelled with 3 or more sublattices and has the {\bf
    SplitPhase} element.  This avoids the need for a meaningless
  disordered DIS\_SIGMA phase.

\item{\bf L:} A BCC or FCC phase with a disordered fraction set can
  have the ``ordered'' part ``as disordered'' subtracted in order to
  allow the modeling of the disordered part separately.  If so the
  value of the {\bf Subtract} attribute should be ``Y''.

  For phases which never disorder, for example the sigma phase, one
  can also have a ``disordered'' part with a single sublattice
  containing the pure elements as endmembers (and possibly some long
  range interaction parameters).  In Thermo-Calc terminology this is
  the ``NEVER'' model.  In order not to subtract the ordered part as
  disordered in XTDB the {\bf Subtract} attribute can be omitted or
  given any other value than ''Y''.

\end{description}
%\end{enumerate}

\subsection{The mathematical expression used in Expr}\label{sec:expr}

The mathematical expressions for $T$ and $P$ used in {\bf TPfun,
  Trange} and {\bf Parameter} are the same as in TDB files.  It is
very limited because some software must calculate first and and second
derivatives with respect to $T, P$ (and constitution).  A more
extended mathematical expression could be allowed for expressions
which are not used for database parameters.  In OC it is allowed to
enter more complex expressions for post processing in batch/macro
files.

In the 1990 definition of the TDB file the type of expression allowed
consists of ``simple terms'' such as:

[signed real number] * [{\bf TPfun Id} ] ** [power] *T** [power] *P**[power]

where [power] is an integer (a negative power must be within
parenthesis).  No spaces allowed in a simple term.  A ``complex term''
is a simple term multiplied with a math function of a simple term,
such as:

[simple term] *LN( [simple term] )

An {\bf Expr} attribute in {\bf TPfun, Trange} or {\bf Parameter}
consists one or more terms.  A positive sign of the first term can be
omitted.
  
The following general math functions are allowed in OC:\\ $\exp(),
\ln(), \log(),$ erf().  Note that $\log()$ and $\ln()$ is the same and
erf() is the error function.  The number of math functions can be
extended.

The following math function is needed for the unary
project:\\ GEIN($\theta$) to calculate
$1.5R\theta+3RT\ln(1-\exp(-\theta/T));$

It is not allowed to group several terms together using parenthesis.

A square root of $T$ is entered as two {\bf TPfun}://
\begin{verbatim}
<TPfun Id="HALFT" Expr="0.5*LN(T);" />
<TPfun Id="SQRT"  Expr="EXP(HALFT);" />
\end{verbatim}

\subsection{The use of wildcards for constituents in parameters}\label{sec:wildcard}

In some parameters a wildcard, ``*'', is used to indicate the
parameter is independent of the constituent in this sublattice.  For
example:
\begin{itemize}
\item A parameter G(sigma,GA:GA:*;0)=100000; means that the Gibbs
  energy of formation of ``phase'' with GA in first and second
  sublattice is independent of the constituent in the third.  If one
  later finds that for example the Gibbs energy for sigma with TA in
  the third sublattice should be just +70000 then, in OpenCalphad, one
  must set G(sigma,GA:GA:TA)=-30000; because OC stores the two
  parameters separately and will add them when calculating.

This may be handled differently in other software and requires attention.

\item The EBEF model, using the same notation for parameters as in
  CEF, replaces a large number of Gibbs endmembers energies by fewer
  {\bf Parameter} elements, representing bond energies, using
  wildcards in one or more sublattices.  The bond energies specify
  constituents in only 2 or 3 sublattices in phases with 3 or more
  sublattices.  Using the {\bf SplitPhase} model and provided there
  two endmember parameters GD(SIGMA,CR) and GD(SIGMA,FE) (or a
  separate disordered phase with these parameters) a parameter
  G(SIGMA,CR:FE:*:*:*) represent the bond energy between CR in first
  and FE in second sublattice, independent on the constituents in the
  other sublattices (note that G(SIGMA,FE:CR:*:*:*) have another
  value).

  Using such parameters (fitted for example to DFT calculated
  endmember energies) the number of parameters can be reduced by more
  than an order of magnitudes and, in addition, extrapolations can be
  improved!

  In order to avoid a phase representing the ``disordered part'' in
  EBEF one can introduce a set of endmember parameters such as
  G(SIGMA,CR:*:*), G(SIGMA,*:CR:*), G(SIGMA,*:*:CR) and
  G(SIGMA,FE:*:*), G(SIGMA,*:FE:*) and G(SIGMA,*:*:FE) (the so called
  ``wild'' model) which is ingenious but likely confusing to students.
  
\end{itemize}

\subsubsection{A missing wildcard in the I2SL model parameters}

An I2SL model may exist with only neutrals in the anion sublattice.
For example the elements C and S can be neutrals in the I2SL model and
their parameters are in Thermo-Calc are written as G(I2SL,C), omitting
the cation sublattice.  In OpenCalphad it is written G(I2SL,*:C) which
may be a bit more consistent as it indicates that the constituent on
the cation sublattice is irrelevant.  It is suggested that the XTDB
format adopts the OC way to specify a constituent in each sublattice
in order to be consistent.  The same for interactions between
neutrals.

Note that an interaction between neutrals with a single specific
cation is forbidden in the I2SL model.

\subsection{Model parameter identifiers, MPID}\label{sec:mpid}

An MPID must start with a letter A-Z and contain letters and digits
and not exceed 8? charactes.  It cannot be abbreviated.  There are
already some defined but there should be an extensive list of future
MPIDs to avoid that different software use the same for (slightly)
different things.  See appendix~\ref{sec:mpid2}.
  
The letter ``\&'', frequently used for mobilities, is forbidden in XML
but it can be replaced by some other character, for example ``@''.
Using ``\&amp;'' seems unnecessary and quite clumsy.

\subsection{CVM and the cluster site model}

These should eventually be included also.

%%%%%%%%%%%%%%%%%%%%%%%%%%%%%%%%%%%%%%%%%%%%%%%%%%%%%%%%%%%%%%%%
% Maybe as appendice?
%\newpage

\section{Points of discussions}

\begin{enumerate}
\item Should mathematical expressions be terminated by a ``;''?  It is
  not necessary because there is a final double quote of the expression
  but it may be nice.

\item The GEIN function is used in the {\bf Parameter} for a pure
  element modeled with several Einstein $\theta$.  Only one of these
  is selected to vary with composition using the LNTH model parameter
  identifier (MPID).

\item Should the sign of Redlich-Kister terms depend on the
  alphabetrical order of the constituents or as the order the
  constituents are written in the parameter {\bf Id}?  Currently the
  TDB file use alphabetical.

%\item Automatic creation of multiple composition sets when reading the
%  XTDB file?  For example in Fe-Ti-C there will normally be an MC
%  carbide with the FCC model parameters.  Could such a composition set
%  have a different name?  It may be useful to provide handling of
%  carbides or ordering in application software.
%
%  But automatic creation of composition sets will require additional
%  features, for example how to identify the different sets?  Using a
%  ``\#digit'' to identify a composition set?  \#1 is the default
%  (first) composition set.  It also requires a way to specify some
%  limits of the constitution of the composition set, for example att
%  the MC carbide have higher fraction of C (or N) than VA.

\end{enumerate}

\section{Summary}

There are certainly many more things to take care of but I think it is
more important to agree on a minimum common XML format which can make
the thermodynamic databases more useful both for experimentalists,
assessments, database maintenance and thermodynamic software in
particular for the development of new models and applications.  We
have to take one step at a time.

%%%%%%%%%%%%%%%%%%%%%%%%%%%%%%%%%%%%%%%%%%%%%%%%%
% Appendices
\newpage
\begin{appendices}
\setcounter{equation}{0}
\renewcommand{\theequation}{A\arabic{equation}}
\setcounter{figure}{0}
\renewcommand{\thefigure}{A\arabic{figure}}

\section{Examples}\label{sec:examples}

A few examples of the XTDB format.  It may not be completely
consistent with the definitions above as I have taken them from my
previous conversion software.

\subsection{Elements and species.}\label{sec:elementexample}

The way to define MQMQA constituents and stoichiometry is tentative.

{\small
\begin{verbatim}
  <Element Id="/-" Refstate="ELECTRON_GAS" Mass="  0.000000E+00" />
  <Element Id="VA" Refstate="VACUUM" Mass="  0.000000E+00" />
  <Element Id="AL" Refstate="FCC_A1" Mass="  2.698200E+01" H298="  4.577300E+03" S298="  2.832200E+01" />
  <Element Id="FE" Refstate="BCC_A2" Mass="  5.584700E+01" H298="  4.489000E+03" S298="  2.728000E+01" />
...
  <Species Id="VA" Stoichiometry="VA" />
  <Species Id="AL" Stoichiometry="AL" />
  <Species Id="FE" Stoichiometry="FE" />
  <Species Id="ALFE" Stoichiometry="AL0.5FE0.5" />
  ...
  <Species Id="LA/F" MQMQA="LA:F 6.0 2.0 2.4" />
  <Species Id="LACS/F" MQMQA="LA,CS:F 9.0 6.0 4.0" />
\end{verbatim}
}

The stoichiometry of the MQMQA species is calculated from the bond
ratios according to the model.  The {\bf MQMQA} attribute for an
``endmember'' also include a factor needed in the configurational
entropy expression.

Note that species cannot be abbreviated as constituents in phases or
parameters and thus one can have species which are abbreviations of
another species.  For example ``O'' and ``O2''.

For phases one can abbreviate its name and thus one cannot have a
phase with a name which is an abbreviation of another phase name.
Phase names can be abbreviated for each part separated by an
underscore, ``\_''.  A phase name ``AL\_X'' is thus an abbreviation of
``AL2\_X''.

\subsection{Defaults, TPfun and Trange}

Using default $T$ limits the function are not much more complex than
in the TDB file.

{\small
\begin{verbatim}
  <Defaults LowT="298.15" HighT="6000" Refid="U.N.Known" Elements="VA /-" />
...
  <TPfun Id="GHSERAL"  >
    <Trange HighT="700" Expr=" -7976.15+137.093038*T-24.3671976*T*LN(T)-.001884662*T**2-8.77664E-07*T**3+74092*T**(-1);" />
    <Trange HighT="933.47" Expr=" -11276.24+223.048446*T-38.5844296*T*LN(T)+.018531982*T**2  -5.764227E-06*T**3+74092*T**(-1);" />
    <Trange HighT="2900" Expr=" -11278.378+188.684153*T-31.748192*T*LN(T)-1.230524E+28*T**(-9);" />
  </TPfun>
...
  <TPfun Id="LFALFE0"  Expr="-104700+30.65*T;" />
  <TPfun Id="LFALFE1"  Expr="+30000-7*T;" />
  <TPfun Id="LFALFE2"  Expr="+32200-17*T;" />
  <TPfun Id="UFALFE"  Expr="-4000+T;" />
  <TPfun Id="GAL3FE"  Expr="+3*UFALFE+9000;" />
  <TPfun Id="GAL2FE2"  Expr="+4*UFALFE;" />
\end{verbatim}
}

\subsection{Phase}\label{sec:phase example}

Entering phases in the XTDB file is a bit more complex but we have get
rid of the TYPE\_DEFINITION.  It is not so nice to use CEF for the
liquid, that is why I think RKM might be a nice option for a liquid
model without sublattices and maybe {\bf Sublattices} can be omitted?

{\small
\begin{verbatim}
 <Phase Id="LIQUID" Configuration="RKM" state="L" >
    <Sublattices Number_of="1"  Ratios="1" >
      <Constituents sublattice="1" list="AL C" />
    </Sublattices>
    <AmendPhase model="LIQ2STATE" />
  </Phase>
...
  <Phase id="A2_BCC" Configuration="CEF" state="S" >
    <Sublattices Number_of="2"  Ratios="1  3" >
      <Constituents sublattice="1" list="AL FE" />
      <Constituents sublattice="2" list="VA" />
    </Sublattices>
    <AmendPhase model="IHJBCC" />
  </Phase>
...
  <Phase id="AL8FE5_D82" Configuration="CEF" state="S" >
    <Sublattices Number_of="2"  Ratios="8  5" >
      <Constituents sublattice="1" list="AL FE" />
      <Constituents sublattice="2" list="AL FE" />
    </Sublattices>
  </Phase>
...
  <Phase id="BCC_4SL" Configuration="CEF" state="S" >
    <Sublattices Number_of="5"  Ratios="0.25  0.25  0.25  0.25  3" >
      <Constituents Sublattice="1" List="AL FE" />
      <Constituents Sublattice="2" List="AL FE" />
      <Constituents Sublattice="3" List="AL FE" />
      <Constituents Sublattice="4" List="AL FE" />
      <Constituents Sublattice="5" List="VA" />
    </Sublattices>
    <SplitPhase Disordered="A2_BCC" Sum="4" Subtract="Y" Refid="09Sun" />
    <AmendPhase model="IHJBCC BCC4Perm" />
  </Phase>

\end{verbatim}
}


\subsection{Parameters}\label{sec:parameter examples}

{\small
\begin{verbatim}
    <Parameter Id="G(A2_BCC,FE:VA;0)"   Expr="+GHSERFE;" Bibref="91Din" />
    <Parameter Id="TC(A2_BCC,FE:VA;0)"   Expr="1043;" Bibref="91Din" />
    <Parameter Id="BMAGN(A2_BCC,FE:VA;0)"   Expr="2.22;" Bibref="91Din" />
    <Parameter Id="G(AL8FE5_D82,AL:AL;0)"   Expr="+13*GALBCC;" Bibref="08Sun" />
...
    <Parameter Id="G(BCC_4SL,AL:AL:FE:FE:VA;0)"   Expr="+GB2ALFE;" Bibref="08Sun" />
    <Parameter id="G(BCC_4SL,AL,FE:*:*:*:VA;1)"   Expr="-634+0.68*T;" Bibref="08Sun" />
\end{verbatim}
}

In the last parameter above the ``wildcard'' or asterisk, ``*'', is
used for species in three of the sublattices and it means that the
parameter is independent of the constituents in these sublattices.
See the discussion in section~\ref{sec:wildcard} how this is treated.

Using the {\bf Parameter2} element the last two parameters above are:

{\small
\begin{verbatim}
    <Parameter2 Id="G(BCC_4SL,AL:AL:FE:FE:VA;0)" Expr="+GB2ALFE;" Bibref="08Sun" MPID="G" Phase="BCC_4SL" >
       <ConstArray Degree="0">
           <Constituent Sublattice="1" Species="AL" />
           <Constituent Sublattice="2" Species="AL" />
           <Constituent Sublattice="3" Species="FE" />
           <Constituent Sublattice="4" Species="FE" />
           <Constituent Sublattice="5" Species="VA" />
       </ConstArray>
    /Parameter2>
...
    <Parameter2 id="G(BCC_4SL,AL,FE:*:*:*:VA;1)" Expr="-634+0.68*T;" Bibref="08Sun" MPID="G" Phase="BCC_4SL" >
       <ConstArray Degree="1">
           <Constituent Sublattice="1" Species="AL" />
           <Constituent Sublattice="1" Species="FE" />
           <Constituent Sublattice="2" Species="*" />
           <Constituent Sublattice="3" Species="*" />
           <Constituent Sublattice="4" Species="*" />
           <Constituent Sublattice="5" Species="VA" />
       </ConstArray>
    /Parameter2>
\end{verbatim}
}

\subsection{Models}\label{sec:modelex}

An extended form of the {\bf Models} should be defined a separate
global library file with the {\bf Id} and full description but the
models used in an XTDB file should appear in a short form as below.

{\small
\begin{verbatim}
  <Models>
    <Magnetic Id="IHJBCC"  MPID1="TC" MPID2="BMAGN" Bibref="82Her" />
    <Magnetic Id="IHJREST"  MPID1="TC" MPID2="BMAGN" Bibref="82Her" />
    <Einstein Id="GLOWTEIN" MPID1="LNTH" bibref="01Qing" /> 
    <Liquid2state Id="LIQ2STATE" MPID1="G2"  MPID2="LNTH" bibref="14Becker" >
       Unified model for the liquid and the amorphous state treated as an Einstein solid
    </Liquid2state>
    <Volume Id="VOLOWP" MPID1="V0"  MPID2="VA" MPID3="VB" bibref="05Lu" />
  ...
  </Models>
  ...
  <ToopModel Phase="Liquid" Constituents="C Fe Cr" />
\end{verbatim}
}

The {\bf ToopModel} or {\bf KohlerModel} elements can appear anywhere
in the XTDB file, even within the {\bf Phase} element or, more likely
together with all the {\bf Parameter} elements from an assessment of
the ternary C-Fe-Cr system in order to simplify editing a large XTDB
file.  Either way may be a bit complicated to handle by the
calculating software reading the XTDB file.

The {\bf SplitPhase} in example~\ref{sec:phase example} must appear
within a {\bf Phase} element for the ordered phase.

\newpage
\section{Proposal of Id for agreed models.}\label{sec:modelapp}

The agreed models need only to added by their idenfiers to the phases.

For models that are not agreed a full description of the model should
be provided in the XTDB file.  Note the model parameter identifiers
may be different in each software.

{\tiny
\begin{verbatim}
  <Models>
    <Magnetic id="IHJBCC"  MPID1="TC" MPID2="BMAGN" anti-ferromagnetic_factor=" -1.00" 
       f_below_TC=" +1-0.905299383*TAO**(-1)-0.153008346*TAO**3-.00680037095*TAO**9-.00153008346*TAO**15;"
       f_above_TC=" -.0641731208*TAO**(-5)-.00203724193*TAO**(-15)-.000427820805*TAO**(-25);" bibref="82Her" > in G=f(TAO)*LN(BETA+1) where TAO=T/TC
    </Magnetic>
    <Magnetic id="IHJREST"  MPID1="TC" MPID2="BMAGN" anti-ferromagnetic_factor=" -3.00" 
       f_below_TC=" +1-0.860338755*TAO**(-1)-0.17449124*TAO**3-.00775516624*TAO**9-.0017449124*TAO**15;"
       f_above_TC=" -.0426902268*TAO**(-5)-.0013552453*TAO**(-15)-.000284601512*TAO**(-25);" bibref="82Her" > in G=f(TAO)*LN(BETA+1) where TAO=T/TC
    </Magnetic>
    <Magnetic id="IHJQX"  MPID1="CT" MPID2="NT" MPID3="BMAGN" anti-ferromagnetic_factor="  0.00" 
       f_below_TC=" +1-0.842849633*TAO**(-1)-0.174242226*TAO**3-.00774409892*TAO**9-.00174242226*TAO**15-.000646538871*TAO**21;"
       f_above_TC=" -.0261039233*TAO**(-7)-.000870130777*TAO**(-21)-.000184262988*TAO**(-35)-6.65916411E-05*TAO**(-49);" bibref="12Xiong" > in G=f(TAO)*LN(BETA+1) where TAO=T/CT or T/NT
    </Magnetic>
    <Einstein id="GLOWTEIN" MPID1="LNTH" bibref="01Qing" > 
       Gibbs energy due to the Einstein low T vibrational entropy model, G=1.5*R*THETA+3*R*T*LN(1-EXP(-THETA/T)).
    </Einstein>
    <Liquid2state id="LIQ2STATE" MPID1="G2"  MPID2="LNTH" bibref="14Becker" >
       Unified model for the liquid and the amorphous state treated as an Einstein solid
    </Liquid2state>
    <Volume id="VOLOWP" MPID1="V0"  MPID2="VA" MPID3="VB" bibref="05Lu" >
       The volume of a phase is described as function of T, P and its constitution.
    </Volume>
    <SplitPhase Disordered=" " Sum=" " Subtract=" " Refid="09Sun" >
       The disordered fractions are summed over the ordered sublattices indicated by Sum.  The Gibbs energy calculated 2 or 3 times as indicated by the CEF_appendix but the Subtract but the configurational entropy only once.
    </SplitPhase>
    <FCC-permutations id="FCC4Perm" bibref="09Sun" >
       Permutations of ordered FCC parameters with the same set of elements are listed only once.
    </FCC-permutations>
    <BCC-permutations id="BCC4Perm" bibref="09Sun" >
       Permutations of ordered BCC parameters with the same set of elements are listed only once.
    </BCC-permutations>
    <EEC id="EEC" bibref="20Sun" >
       Equi-Entropy Criterion means that solid phases with higher entropy that the liquid phase must not be stable.
    </EEC>
  </Models>
\end{verbatim}
}

\newpage
\section{Proposal of initial set of Model Parameter Identifiers, MPID}\label{sec:mpid2}

The 32 model parameter identifiers, MPID, defined in OC as shown in
Table~\ref{tab:mpis}.  In OC a parameter for a disordred part of a
{\bf SplitPhase} use the same phase name but the MPID has a suffix
``D''.

\begin{table}[!h]
  \caption{Current set of model parameter identifiers in OC.  For each
    parameter it is indicated if it can depend on $T$, $P$ or have an
    extra constituent specification.  Most of them have no associated
    code.}\label{tab:mpis}

{\small
\begin{verbatim}
Indx Ident T P Specification                Status Note
   1 G     T P                                   0 Gibbs Energy
   2 TC    - P                                   2 Combined Curie/Neel T
   3 BMAG  - -                                   1 Average Bohr magneton number
   4 CTA   - P                                   2 Curie temperature
   5 NTA   - P                                   2 Neel temperature
   6 IBM   - P &<constituent#sublattice>;       12 Individual Bohr magneton number
   7 LNTH  - P                                   2 Einstein temperature
   8 V0    - -                                   1 Volume at T0, P0
   9 VA    T -                                   4 Thermal expansion
  10 VB    T P                                   0 Bulk modulus
  11 VC    T P                                   0 Alternative volume parameter
  12 VS    T P                                   0 Diffusion volume parameter
  13 MQ    T P &<constituent#sublattice>;       10 Mobility activation energy
  14 MF    T P &<constituent#sublattice>;       10 RT*ln(mobility freq.fact.)
  15 MG    T P &<constituent#sublattice>;       10 Magnetic mobility factor
  16 G2    T P                                   0 Liquid two state parameter
  17 THT2  - P                                   2 Smooth step function T
  18 DCP2  - P                                   2 Smooth step function value
  19 LPX   T P                                   0 Lattice param X axis
  20 LPY   T P                                   0 Lattice param Y axis
  21 LPZ   T P                                   0 Lattice param Z axis
  22 LPTH  T P                                   0 Lattice angle TH
  23 EC11  T P                                   0 Elastic const C11
  24 EC12  T P                                   0 Elastic const C12
  25 EC44  T P                                   0 Elastic const C44
  26 UQT   T P &<constituent#sublattice>;       10 UNIQUAC residual parameter (OC)
  27 RHO   T P                                   0 Electric resistivity
  28 VISC  T P                                   0 Viscosity
  29 LAMB  T P                                   0 Thermal conductivity
  30 HMVA  T P                                   0 Enthalpy of vacancy formation (MatCalc)
  31 TSCH  - P                                   2 Schottky anomaly T (OC)
  32 CSCH  - P                                   2 Schottky anomaly Cp/R. (OC)
  33 NONE  T P                                   0 Unused
\end{verbatim}
  }
\end{table}
 
\end{appendices}

\end{document}


